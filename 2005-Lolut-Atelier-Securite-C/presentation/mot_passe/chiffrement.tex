\section{Chiffrement et signature}

\subsection{Algorithmes de chiffrement}

\begin{frame}
  \frametitle{Algorithmes de chiffrement}

  \begin{columns}
    \begin{column}{1.0\textwidth}<1->
      \begin{itemize}
      \item<1-> Anc�tres : XOR, Cesar, etc.
      \item<2-> Sym�triques : RC5, DES, AES, etc.
      \item<3-> Asym�triques : RSA, ElGamal, etc.
      \end{itemize}
    \end{column}
  \end{columns}
\end{frame}

\subsection{Algorithmes de signature}

\begin{frame}
  \frametitle{Algorithmes de signature}

  \begin{columns}
    \begin{column}{0.9\textwidth}<1->
      \begin{itemize}
      
      \item<1-> Un hash est un condensat d'une longueur fixe d'une cha�ne
      de caract�re ou d'un fichier. On s'en sert pour signer un document
      �lectronique (v�rifier qu'il n'a pas �t� modifi�). En cryptographie,
      le but rechercher est qu'il soit difficile de revenir au mot de passe
      en partant du hash.

      \item<2-> Anc�tres : CRC32.

      \item<3-> D�conseill�s : MD5 et SHA-1.
      
      \item<4-> Modernes : Whirlpool et Tiger (orient�s s�curit�).
      \end{itemize}
    \end{column}
  \end{columns}
\end{frame}


