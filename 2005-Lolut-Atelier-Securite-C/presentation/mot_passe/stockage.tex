\section{Les mots de passe}

\subsection{Stockage du mot de passe}

\begin{frame}
  \frametitle{Stockage du mot de passe}

  \begin{columns}
    \begin{column}{1.0\textwidth}<1->
      \begin{itemize}
      \item<1-> Stockage en clair. Exemple : Gaim.
      \item<2-> Codage standard comme base64. Exemples : evolution, ncftp, yafc.
      \item<3-> Codage maison. Exemple : gftp.
      \item<4-> Stockage sous forme de hash. Exemple : mot de passe Unix.
      \item<5-> Vrai chiffrement. Exemple : Firefox (si le mot de passe
      principal est d�fini) et KWallet (sous KDE).
      \end{itemize}
    \end{column}
  \end{columns}
\end{frame}

\subsection{Transfert sur le r�seau}

\begin{frame}
  \frametitle{Mots de passe circulant sur le r�seau}

  \begin{columns}
    \begin{column}{1.0\textwidth}<1->
      \begin{itemize}
      \item<1-> De nombreux protocoles de communications envoyent les mots
      de passe en clair: telnet, POP3, FTP, HTTP Basic, etc.
      \item<2-> On peut encapsuler ces protocoles dans un tunnel s�curis�
      tel que SSH.
      \item<3-> Ou bien utiliser des protocoles plus s�rs tel que POP3 par SSL,
      HTTPS, SFTP, SSH, etc.
      \end{itemize}
    \end{column}
  \end{columns}
\end{frame}


